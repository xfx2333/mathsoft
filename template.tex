\documentclass{ctexart}

\usepackage{graphicx}
\usepackage{amsmath}

\title{作业一: 叙述并证明带皮亚诺余项的泰勒定理}


\author{嵇敏君 \\信息与计算科学 3200103322 }

\begin{document}

\maketitle


\section{问题描述}
问题叙述如下: 若函数f在点$x_0$存在直至n阶导数,则有
\begin{equation}
  f(x)=T_n(x)+o((x-x_0)^n)\label{pythagorean}
\end{equation}
,即
\begin{equation}
  f(x)=f(x_0)+f'(x_0)(x-x_0)+\frac{f''(x_0)}{2!}(x-x_0)^2+...+\frac{f^{(n)}x_0)}{n!}(x-x_0)^n\label{pythagorean}
\end{equation}


\section{证明}
设
\begin{equation}
  R_n(x)=f(x)-T_n(x),Q(x)=(x-x_0)^n,\notag
\end{equation}
现在只要证
\begin{equation}
  \lim_{x \rightarrow x_0}\frac{R_n(x)}{Q_n(x)}=0\label{pythagorean}
\end{equation}
有关系式
\begin{equation}
  f^{(k)}(x_0)=T^{(k)}_n(x_0),k=0,1,2,...,n.\label{pythagorean}
\end{equation}
可知,
\begin{equation}
  R_n(x_0)=R'_n(x_0)=...=R^{(n)}_n(x_0)=0,\label{pythagorean}
\end{equation}
并易知
\begin{equation}
  Q_n(x_0)=Q'_n(x_0)=...=Q^{(n-1)}_n(x_0)=0,Q^{(n)}_n=n!.\label{pythagorean}
\end{equation}
因为$f^{(n)}(x_0)$存在,所以在点$x_0$的某邻域$U(x_0)$上f存在n-1阶导函数.于是,当$x \in U^o(x_0)$且$x \rightarrow x_0$时,允许接连你使用洛必达法则n-1次,得到
\begin{align}
  \lim_{x \rightarrow x_0}\frac{R_n(x)}{Q_n(x)} = & \lim_{x \rightarrow x_0}\frac{R'_n(x)}{Q'_n(x)}=...=\lim_{x \rightarrow x_0}\frac{R^{(n-1)}_n(x)}{Q^{(n-1)}_n(x)} \notag \\
= & \lim_{x \rightarrow x_0}\frac{f^{(n-1)}(x)-f^{(n-1)}(x_0)-f^{(n)}(x_0)(x-x_0)}{n(n-1)...2(x-x_0)} \notag \\
= & \frac{1}{n!}\lim_{x \rightarrow x_0}[\frac{f^{(n-1)}(x)-f^{(n-1)}(x_0)}{x-x_0}-f^{(n)}(x_0)] \notag \\
= & 0 \notag
\end{align}


\end{document}
