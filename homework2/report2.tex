\documentclass{ctexart}


\title{作业二: shell文件下的分支结构与循环结构}


\author{嵇敏君 \\信息与计算科学 3200103322 }

\begin{document}

\maketitle

shell文件下的分支结构与循环结构同python,C++等的结构有较大的区别,下面便大致区分一下shell文件的分支结构与循环结构同python,C++的区别
\section{分支结构}
分支结构主要是if语句与case语句
\subsection{if语句}
python的if语句结构体如下:
\begin{verbatim}
if condition
  statements
elif condition
  statements
else
  statements
\end{verbatim}
而在shell文件下,if结构体变成了:
\begin{verbatim}
if condition
then
  statements
elif condition
then
  statements
else
  statements
fi
\end{verbatim}
这里
\begin{verbatim}
if/elif condition
then
\end{verbatim}
也可写成
\begin{verbatim}
if/elif condition; then
\end{verbatim}
可以看出,在shell文件下,if语句在判断之后,\verb|if|与\verb|elif|的后面都接上了一个\verb|then|,同时,在整个if语句结束之后,shell文件最后一行还需要加上一个\verb|fi|,而且在shell文件中,condition的判断需要用[]
接着,我们用书本35页中的一个案例来测试一下,测试代码如下:
\begin{verbatim}
#!/bin/bash

echo "Is it morning? Please answer yes or no"
read timeofday
if [ $timeofday = "yes" ]
then
  echo "Good morning"
elif [ $timeofday = "no" ]; then
  echo "Good afternoon"
else
  echo "Sorry,$timeofday not recognized.Enter yes or no"
  exit1
fi

exit 0
\end{verbatim}
这里我们要注意条件判断时,\verb|$timeofday="yes"|与\verb|"$timeofday"="yes"|的区别
\subsection{case语句}
在python中,case语句的结构体为:
\begin{verbatim}
switch (variable){
case pattern1: statements;
case pattern2: statements;
...
case patternn: statements;
default: statemens;}
\end{verbatim}
case结构体如下:
\begin{verbatim}
case variable in
  pattern [ | pattern] ...) statements;;
  pattern [ | pattern] ...) statements;;
  ...
esac
\end{verbatim}
可以看到,shell文件的case语句与python中的语句差异较大,python中case会与switch一起使用,每个statements后都有一个;,结尾处有一个default来表示其他所有情况.而在shell文件中,case作为开头与in一起使用,每个statements后都有两个;(;;),同时结尾处用了一个esac(case反过来写)表示结束
下面我们用书本中的几个案例来体会一下case
case1:
\begin{verbatim}
#!/bin/bash

echo "Is it morning? Please answer yes or no"
read timeofday

case "$timeofday" in
    yes) echo "Good morning";;
    no ) echo "Good afternoon";;
    y  ) echo "Good morning";;
    n  ) echo "Good afternoon";;
    *  ) echo "Sorry, answer not recognized";;
esac

exit 0
\end{verbatim}
可以看出,在shell文件中,用*来表示其他所有的情况,但是,这样子不能尽可能多的作出正确的判断,如对于大小写的判断就不够灵敏,于是便有如下两种方法
case2:
\begin{verbatim}
#!/bin/bash

echo "Is it morning? Please answer yes or no"
read timeofday

case "$timeofday" in
    yes | y | Yes | YES) echo "Good morning";;
    n* | N* ) echo "Good afternoon";;
    *  ) echo "Sorry, answer not recognized";;
esac

exit 0
\end{verbatim}
case3:
\begin{verbatim}
#!/bin/bash

echo "Is it morning? Please answer yes or no"
read timeofday

case "$timeofday" in
    yes | y | Yes | YES) 
           echo "Good morning"
           echo "Up bright and early this morning";;
    [nN]* ) 
           echo "Good afternoon";;
    *) 
           echo "Sorry, answer not recognized"
           echo "Please answer yes or no"
           exit 1;;
esac

exit 0
\end{verbatim}
\section{循环结构}

\subsection{for语句}
在shell文件中,由于输入与进行判断的都是字符串,而不是像python中的一个个数字,故for语句也只能进行字符串的替换,结构体如下:
\begin{verbatim}
for ... in srt1 str2 str3
do
  statements
done
\end{verbatim}
相比与python,可以看到我们在for语句后面多了一个\verb|do|,在结束时加上了一个\verb|done|
同样,我们用书本37页的案例进行测试,测试代码如下:
\begin{verbatim}
#!/bin/bash

for foo in bar fud 43
do
  echo $foo
done
exit 0
\end{verbatim}
输出的结果为:
\begin{verbatim}
bar
fud
43
\end{verbatim}
但是,若我们将代码换成:
\begin{verbatim}
for foo in "bar fud" 43
do
  echo $foo
done
exit 0
\end{verbatim}
此时输出为:
\begin{verbatim}
bar fud
43
\end{verbatim}
可以知,当两个字符串被一个""包裹在一起时,他们便构成了一个新的字符串,for语句是将in后面的所有字符串一个一个替换到foo中去,然后执行do后面的语句
\subsection{while语句}
shell文件中的while结构体与python中相差不大,就是最判断结束时多了一个\verb|done|,具体结构体如下:
\begin{verbatim}
while condition do
  statements
done
\end{verbatim}
我们用39页的一个代码测试:
\begin{verbatim}
#!/bin/bash

echo "Enter password"
read trythis

while [ "$trythis" != "secret" ]; do
  echo "Sorry, try again"
  read trythis
done
echo "Pass"
exit 0
\end{verbatim}
\subsection{until语句}
until语句的情况与while类似,只不过一个是当满足condition时,执行statements(while),一个是当满足condition时结束循环(until),until结构体如下:
\begin{verbatim}
until condition
do
  statements
done
\end{verbatim}
同样,用书本40页中的代码进行测试:
\begin{verbatim}
#!/bin/bash

until who |grep "$1" > /dev/null
do
  sleep 60
done

# now ring the bell and announce the expected user.

echo -e '\a'
echo ***** $1 has just logged in *****

exit 0
\end{verbatim}
\end{document}
