\documentclass{ctexart}

\title{作业三:规划Linux的工作环境}


\author{嵇敏君 \\信息与计算科学 3200103322 }

\begin{document}

\maketitle

\section{介绍我的Linux}
输入命令\cite{王唯融2017一种}
\begin{verbatim}
cat /etc/os-release
\end{verbatim}
得到了
\begin{verbatim}
NAME="Ubuntu"
VERSION="16.04.7 LTS (Xenial Xerus)"
\end{verbatim}
于是Linux发行版名称与版本号便可得出
\section{一些调整}
  我对Linux系统没有做过什么调整,差不多就是安装好Linux虚拟机之后就没有改动过,自己也已经习惯了它的原始配置(一方面是自己懒,另一方面也是因为自己确实啥也不懂,不敢乱改动,电脑小白瑟瑟发抖)

    然后安装的软件的话,大部分也都是按着老师在学在浙大上上传的视频里面,在\verb|Synaptic Packages Manager|里安装了一些内容,如emacs,vim(好像就没用过),gcc等,除了这些的话,我在Linux里还安装了一个VSCode.

    其他的额外的配置工作的话,可能就是安装了一个中文输入法,方便打字什么的.
\section{规划下一步}
\subsection{预计半年内的使用场合}
未来半年内(其实也就是差不多下个学期),由于我是20级信息与计算科学的,于是下学期有王何宇老师的数值分析,我相信在这门课上我肯定会使用到Linux环境,估计也就只有这个机会这个场合能够使用到Linux环境了.
\subsection{分析}
    我觉得我现在的工作环境还是比较稳定的,因为这个环境是从去年的数据结构与算法就保留下来的,我已经习惯了这个环境,而且这个环境已经经历过数据结构这门课的考验了,于是它肯定能够满足我的未来需求.

    不过,就在前几天,我的terminal突然打不开了,在网上找了很多办法也没有解决问题,不过好在当初下了一个VSCode,现在只能凭着这里面的终端暂时顶用一下,等这门课结束后我再去考虑一下是否要进行重装,不过我已经熟悉了这个环境,所以现在还是比较纠结中.
\section{安全性与稳定性保证}
老师上课说过,为了防止某天电脑发生事故(比如,丢了,被车压碎了,掉进马桶里被冲走了之类的),我们需要对所有的文件进行一个备份,主要有两种方法
\subsection{github YYDS}
第一种肯定是将我们的所有代码,结果,文献,配置都上传到github上去,用\verb|git add|,\verb|git commit|,\verb|git push|等操作便可以把文件上传到github上,不过github着实有点看脸,我这种脸黑的就常常登不上去.
\subsection{各种“云”}
除了github,我们也可以把代码保存在各种各样的“云”(阿里云,坚果云等等等等)之中,不过大部分“云”都需要一点点的氪金,像我们这种尚未完全经济独立的大学生不太承受得起,当然也有一些免费的,像坚果云便可以免费使用
\subsection{第三种看起来安全但又不那么安全的方法}
或许有人想过把代码copy一份保存在u盘里,老师上课虽然说不太好,说u盘出事故的可能性比电脑还大,但我觉得吧,如果用u盘备份好了之后把u盘放到一专用的存放东西的地方并做好标签的话,一定程度上也是能够规避掉一些风险的.

\bibliographystyle{plain}
\bibliography{jiminjun}

\end{document}

